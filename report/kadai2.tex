\documentclass[]{jsarticle}
\usepackage[dvipdfm]{graphicx}
\usepackage{here}
\usepackage{multicol}
\usepackage{comment}
\usepackage{moreverb}
\usepackage{ascmac,here,txfonts,txfonts}
\usepackage{color}
\usepackage{listings,jlisting}
\renewcommand{\lstlistingname}{リスト}
\lstset{
  breaklines = true,
  language=Java,
  basicstyle=\ttfamily\scriptsize,
  commentstyle={\itshape \color[cmyk]{1,0.4,1,0}},
  classoffset=1,
  keywordstyle={\bfseries \color[cmyk]{0,1,0,0}},
  frame=tRBl,
  framesep=5pt,
  showstringspaces=false,
  numbers=left,
  stepnumber=1,
  numberstyle=\tiny,
  tabsize=2,
}

\title{\LARGE {数値計算法 課題2(常微分方程式)}}
\author{\large {ME1507 芝田 将}}

\begin{document}

\maketitle

\section{ロジスティック方程式の初期値問題$y'(t)=(1-y)y ~~~ (0 \ge t \ge 2),~~ y(0)=0.1$に関する次の問(1-1)から(1-3)について以下の4つの各解法を使って答えよ。}

\begin{enumerate}
\item オイラー法
\item ホイン法
\item 4次のルンゲクッタ法
\item 2次アダムス・バッシュフォース法
\end{enumerate}

\subsection{$h=\Delta t = 0.1, 0.01$の場合に$y(2)$の近似値を求めよ。}

各解法で近似値を求めるプログラムをリスト\ref{kadai1-source}に示す。

\lstinputlisting[caption=課題2プログラム,label=kadai2-source]
{../src/kadai2.c}


実行結果を以下に示す。

\begin{lstlisting}[caption=実行結果,label=kadai2-result]
$ ./a.out
Please input a h value (0.1 or 0.01).
h > 0.1
オイラー法による解: 0.438414
ホイン法による解: 0.450483
ルンゲクッタ法による解: 0.450853

$ ./a.out
Please input a h value (0.1 or 0.01).
h > 0.01
オイラー法による解: 0.449601
ホイン法による解: 0.450849
ルンゲクッタ法による解: 0.450853
\end{lstlisting}



\subsection{結果を考察せよ}

\subsection{各解法による近似解を比較出来るようにグラフで表示せよ。}


\section{2次アダムス・バッシュフォース法を導け}

\section{2次のアダムス・モルトン法(台形法)が2次の方法であることを示せ}

\section{P110 第5章 演習問題7を解け}



\end{document}