\documentclass[]{jsarticle}
\usepackage[dvipdfm]{graphicx}
\usepackage{here}
\usepackage{multicol}
\usepackage{comment}
\usepackage{moreverb}
\usepackage{ascmac,here,txfonts,txfonts}
\usepackage{color}
\usepackage{listings,jlisting}
\renewcommand{\lstlistingname}{リスト}
\lstset{
  breaklines = true,
  language=Java,
  basicstyle=\ttfamily\scriptsize,
  commentstyle={\itshape \color[cmyk]{1,0.4,1,0}},
  classoffset=1,
  keywordstyle={\bfseries \color[cmyk]{0,1,0,0}},
  frame=tRBl,
  framesep=5pt,
  showstringspaces=false,
  numbers=left,
  stepnumber=1,
  numberstyle=\tiny,
  tabsize=2,
}

\title{\LARGE {数値計算法 課題1(誤差)}}
\author{\large {ME1507 芝田 将}}

\begin{document}

\maketitle

\section{2次方程式$ax^{2}+bx+c=0$の解について(桁落ち,情報落ち)}

\subsection{公式をそのまま使って解を計算すると、桁落ち・情報落ちが発生することがある。桁落ち・情報落ちを回避する方法と、回避できる理由を述べよ。}

$b>0$の場合、分子の$-b + \sqrt{b^{2}-4ac}$の計算で、桁落ちが起きる可能性がある。
そこで$x_{1}$の計算では$x2$を用いることで以下のように書き換えると桁落ちを回避することが出来る。

\begin{eqnarray}
x_{1} &=& \frac{c}{ax_{2}} \\
x_{2} &=& \frac{-b - \sqrt{b^{2}-4ac}}{2a}
\end{eqnarray}

また$b<0$の場合、分子の$-b - \sqrt{b^{2}-4ac}$の計算で、桁落ちが起きる可能性がある。
そこで$x_{1}$の計算では$x2$を用いることで以下のように書き換えると桁落ちを回避することが出来る。

\begin{eqnarray}
x_{1} &=& \frac{-b + \sqrt{b^{2}-4ac}}{2a} \\
x_{2} &=& \frac{c}{ax_{1}}
\end{eqnarray}


\subsection{次の入出力を持つプログラムを作成するためのアルゴリズムもしくはフローチャートを作成せよ。}

\begin{itemize}
\item 解の公式をそのまま使って計算した例
\begin{enumerate}
\item 判別式D (=$b^{2}-4ac$)を計算し、正負を判断する。
\item 正なら解の公式の計算結果をそのまま表示。
\item 負なら実数部と虚部を計算し表示。
\end{enumerate}

\item 桁落ち・情報落ちを回避する方法で計算した例
\begin{enumerate}
\item 解の公式を以下のように変形して計算
\begin{itemize}
\item $b>0$の場合
\begin{eqnarray}
x_{1} &=& \frac{c}{ax_{2}} \\
x_{2} &=& \frac{-b - \sqrt{b^{2}-4ac}}{2a}
\end{eqnarray}
\item $b<0$の場合
\begin{eqnarray}
x_{1} &=& \frac{-b + \sqrt{b^{2}-4ac}}{2a} \\
x_{2} &=& \frac{c}{ax_{1}}
\end{eqnarray}
\end{itemize}

\end{enumerate}

\end{itemize}


\subsection{プログラムを作成せよ}

プログラムをリスト\ref{kadai1-source}に示す。

\lstinputlisting[caption=課題1プログラム,label=kadai1-source]
{../src/kadai1-1.c}


\subsection{結果を考察せよ}

実行結果をリスト\ref{kadai1-result}に示す。

\begin{lstlisting}[caption=実行結果,label=kadai1-result]
$ ./a.out
係数を入力して下さい。
a > 0.01
b > 100
c > 0.01

そのまま計算した場合
2次方程式の解: x = 0.000000, -10000.000000です

桁落ち・情報落ちを回避する方法で計算した解
2次方程式の解: x = -0.000100, -10000.000000です
\end{lstlisting}

異なる2つの実数解を得るために$a=0.01, b=100, c=0.01$を入力値として与えた。
有理化を行うことでより正確な値となっていることが確認できる。
分子の有理化を行なった式を適用し計算することで、情報落ちを回避できたと考えられる。


\section{オイラー定数$\gamma:=\lim_{n \to \infty} \left(\sum_{i=1}^{n} \frac{1}{i} - \log_{e} n\right)=0.5772156...$の近似値について(誤差の累積)}

\subsection{次の入出力を持つプログラムを作成せよ}

\begin{itemize}
\item 入力: $n$
\item 出力:
\begin{itemize}
\item $\frac{1}{1}$から順に$\frac{1}{n}$まで加えた後、$\log_{e} n$の値を引いて得られた近似値
\item $\frac{1}{n}$から順に$\frac{1}{1}$まで加えた後、$\log_{e} n$の値を引いて得られた近似値
\end{itemize}
\end{itemize}

作成したプログラムをリスト\ref{kadai2-source}に示す。

\lstinputlisting[caption=課題2プログラム,label=kadai2-source]
{../src/kadai1-2.c}

\subsection{$n$を変えてプログラムを実行し、誤差の視点から結果を考察せよ}

$n$を100, 10000, 1000000と変化させた際の実行結果をリスト\ref{kadai2-result}に示す。

\begin{lstlisting}[caption=実行結果,label=kadai2-result]
$ ./a.out
nを入力して下さい。
n > 100
1/1から1/n: 0.582208
1/nから1/1: 0.582207

$ ./a.out
nを入力して下さい。
n > 1000
1/1から1/n: 0.577723
1/nから1/1: 0.577716

$ ./a.out
nを入力して下さい。
n > 10000
1/1から1/n: 0.577273
1/nから1/1: 0.577264

$ ./a.out
nを入力して下さい。
n > 1000000
1/1から1/n: 0.541847
1/nから1/1: 0.577141
\end{lstlisting}

入力値として$n=100, n=1000, n=10000, n=1000000$を与えた。
$n$が大きくなるにつれ、2つの計算方法に現れる差が大きくなっていることが確認できる。
情報落ちが発生したと考えられる。

\end{document}